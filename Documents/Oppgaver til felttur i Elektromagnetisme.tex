
\documentclass[twoside,utf8]{article}
\usepackage{lipsum} % Package to generate dummy text throughout this template
\usepackage{comment}
\usepackage{amsmath, amssymb}
\usepackage{eulervm}
\usepackage{tensor}
\usepackage{calc}
\usepackage[utf8]{inputenc}
%\usepackage{mathpazo}
%\usepackage[math]{anttor}
%\usepackage{cmbright}
%\usepackage{mathastext}

\usepackage[usenames,dvipsnames]{xcolor}
\usepackage{graphicx}
% \usepackage[T1]{fontenc} % Use 8-bit encoding that has 256 glyphs
\linespread{1.1} % Line spacing - Palatino needs more space between lines
% \usepackage{microtype} % Slightly tweak font spacing for aesthetics
\usepackage[hmarginratio=1:1,top=32mm,columnsep=20pt]{geometry} % Document margins
\usepackage{multicol} % Used for the two-column layout of the document
\usepackage[hang, small,labelfont=bf,up,textfont=it,up]{caption} % Custom captions under/above floats in tables or figures
\usepackage{booktabs} % Horizontal rules in tables
\usepackage{hyperref} % For hyperlinks in the PDF
\usepackage{titlesec} % Allows customization of titles
\usepackage{slashed}
\usepackage{simplewick}
\usepackage[force]{feynmp-auto}

\renewcommand{\abstractnamefont}{\normalfont\bfseries} % Set the "Abstract" text to bold
\renewcommand{\abstracttextfont}{\normalfont\small\itshape} % Set the abstract itself to small italic text
\renewcommand\thesection{\Roman{section}} % Roman numerals for the sections
\renewcommand\thesubsection{\Roman{subsection}} % Roman numerals for subsections
\titleformat{\section}[block]{\large\scshape\centering\bfseries}{\thesection.}{1em}{} % Change the look of the section titles
\titleformat{\subsection}[block]{\scshape\bfseries}{\thesubsection.}{1em}{} % Change the look of the section titles

\newcommand{\EQU}[1] { \begin{equation*} \begin{split} #1 \end{split} \end{equation*} }
\newcommand{\EQUn}[1] { \begin{equation} \begin{split} #1 \end{split} \end{equation} }
\newcommand{\PAR}[2]{ \frac{\partial #1}{\partial #2}}
\newcommand{\ket}[1] { |#1\rangle }
\newcommand{\expe}[1]{ \langle #1 \rangle }
\newcommand{\bra}[1] { \langle #1 | }
\newcommand{\braket}[2] { \langle #1 | #2 \rangle }
\newcommand{\creation   }[1]{ a_\mathbf{ #1 }^\dagger }
\newcommand{\destruction}[1]{ a_\mathbf{ #1 } }



%----------------------------------------------------------------------------------------
%	TITLE SECTION
%----------------------------------------------------------------------------------------

\title{\vspace{-15mm}\fontsize{24pt}{10pt}\selectfont Felttur 2016 \\ \textbf{Elektromagnetisme} } % Article title

\author{
\large
\textsc{August Geelmuyden}\\[2mm] % Your name
\normalsize Universitetet i Oslo \\ % Your institution
\vspace{-5mm}
}
\date{}

%-------------------------------------------------------------------------------

\begin{document}

\maketitle % Insert title




\part*{Teori}

\section{Påvirkning uten berøring}
Når to objekter påvirker hverandre uten å være i berøring er det ofte naturlig å introdusere konseptet {\it felt}. Feltets rolle er å formidle påvirkningen fra det ene objektet til det andre. På den måten kan man unngå den litt problematiske ideen om at romlig separerte objekter påvirker hverandre ved heller å tenke på hendelsen som at objektet påvirker feltet, feltet brer seg utover og feltet påvirker det andre objektet. Det fine med dette synspunktet er at ideen om påvirkning som et lokalt konsept er ivaretatt.

Et eksempel på en situasjon der det er naturlig å introdusere et felt er for å forklare samspillet mellom elektriske ladninger. Vi er her interessert i kreftene de elektriske ladningene utfører på hverandre, noe som betyr at vi bør studere et {\it vektorfelt}. Feltet for elektrisitet kalles {\it elektrisk felt} og betegnes ofte ved
$\mathbf{E}=\mathbf{E}(t,\mathbf{r})$. Kraften det elektriske feltet utøver på et punktlegeme med ladning $q$  er gitt ved $\mathbf{F}=q\mathbf{E}$.
Likningen $\mathbf{F}=q\mathbf{E}$ er matematisk ekvivalent med Newtons andre lov $\mathbf{F}=m\mathbf{a}$, bare med den ekstra muligheten at $m < 0$. Et vektorfelt som på mange måter likner det elektriske feltet er {\it tyngdefeltet}. Tyngdefeltet til en punktmasse $m$ kan tenkes på som tyngdeakselerasjonen massen gir opphav til. I analogi til likningen
$$
|\mathbf{g}| \propto \frac{m}{r^2}
$$
kan vi altså skrive
$$
|\mathbf{E}| \propto \frac{q}{r^2}.
$$
Dette er loven den franske fysikeren Charles Augustin de Coulomb oppdaget i 1784. Loven, som i ettertid har blitt kjent under navnet {\it Coulombs lov}, kan også skrives
$$
\mathbf{F} = \frac{1}{4 \pi \varepsilon_0 } \frac{q_1 q_2}{r^3}\mathbf{r}
$$
der $\mathbf{F}$ er kraften mellom to punktpartikler med ladning  $q_1$ og $q_2$ som befinner seg i en avstand $r$ fra hvarandre. Proporsjonalitetskonstanten, $k_e=1/4 \pi \varepsilon_0$, kalles {\it Coulombs konstant}. På samme måte som tyngdefeltet har et potensial, kan vi konstruere et elektrisk potensial slik at $\mathbf{E}=-\nabla V$. Størrelsen på det elektriske potensialet kalles {\it spenning} og måles i {\it Volt}. Fra Coulombs lov finner vi at
$$
V = \frac{1}{4\pi \varepsilon_0} \frac{q}{r}
$$
for en punktladning $q$. Ha imidlertid i tankene at det elektriske feltets direkte avhengighet av det elektriske potensialet ikke lenger stemmer i nærvær av et magnetisk felt. Da må også et vektorpotensial introduseres.



\section{Maxwells lover}
\subsection{Gauss lov}

Fra definisjonen av det elektriske feltet følger det at den elektriske fluksen ut av et volum må være proporsjonal med ladningen volumet inneholder. Faktisk er
$$
\iint _{\partial V} \mathbf{E}\cdot d\mathbf{A} = \frac{q_{int}}{\varepsilon_0}
$$
der $\partial V$ er overflaten til volumet, $\mathbf{E}$ er det elektriske feltet, $d\mathbf{A}$ er en uendelig liten bit $dA$ av $\partial V$ med retning vinkelrett, ut av volumet på den relevante delen av $\partial V$. $q_{int}$ er den totale ladningen volumet inneholder og $\varepsilon_0$ er en proporsjonalitetskonstant kalt {\it vakuumpermittiviteten}.


  Som nevnt er det elektriske feltet og tyngdefeltet av matematisk lik form. Det er derfor ikke overraskende at også tyngefeltet har en {\it Gauss lov}:
  $$
  \iint _{\partial V} \mathbf{g}\cdot d\mathbf{A} = -4 \pi G m
  $$
  der $\mathbf{g}$ er tyngdeakselerasjonen, $G$ er gravitasjonskonstanten og $m$ er massen inneholdt i volumet.



  Gauss lov er et veldig nyttig verktøy for å beregne det elektriske feltet til et ladet legeme. Eksempelvis kan vi finne det elektriske feltet til et kuleskall med radius $R$ og uniform ladning $q$ en avstand $r>R$ fra kulen ved å velge $\partial V$ til å være et kuleskall av radius $r$. Det betyr at
  $$
  \iint _{\partial V} \mathbf{E}\cdot d\mathbf{A}
  =
  |\mathbf{E}| \iint _{\partial V} d\mathbf{A}
  =
  |\mathbf{E}| 4\pi r^2
  =
  4 \pi r^2 |\mathbf{E}|
  =
  \frac{q}{\varepsilon_0}.
  $$
  Med andre ord setter et uniformt ladet kuleskall opp et elektrisk felt som er identisk med feltet fra en punktladning i sentrum av kulen, altså
  $$
  \mathbf{E} = \frac{1}{4\pi \varepsilon_0} \frac{q}{r^2}\hat{\mathbf{r}}.
  $$




  Ved å introdusere konseptet om ladnings{\it tetthet} $\rho$ kan vi relatere et legemes totale ladning $q$ med dets ladningstetthet ved å observere at
  $$
  q = \iiint_V \rho dV.
  $$
  Gauss lov kan dermed skrives
  $$
  \iint_{\partial V} \mathbf{E}\cdot d\mathbf{A} = \iiint_V \frac{\rho}{\varepsilon} dV,
  $$
  som ved å benytte {\it divergensteoremet} gjør at vi kan skrive
  $$
  \iiint_{V} \nabla \cdot \mathbf{E} dV = \iiint_V \frac{\rho}{\varepsilon} dV.
  $$
  Siden begge sider integreres over det samme kan vi samle de to integralene til ett:
  $$
  \iiint_{V} \left( \nabla \cdot \mathbf{E} - \frac{\rho}{\varepsilon} \right)dV=0.
  $$
  Legg merke til at dette skal stemme for absolutt alle volumer $V$. Det må bety at dette egentlig er en egenskap som angår objektene inni integralet. Altså
  $$
  \nabla \cdot \mathbf{E} = \frac{\rho}{\varepsilon}.
  $$
  Dette er Gauss lov på differensialform.



\subsection{Gauss lov for magneter}
Ved å resirkulere argumentene i begynnelsen av denne teksten bør det, siden magnetisme er en form for påvirkning uten berøring, være mulig å forklare magnetisme ved hjelp av et {\it magnetisk felt} $\mathbf{B}$. I likhet med det elektriske feltet ble det magnetiske feltets $1/r^2$-avhengighet bekreftet på slutten av 1700-tallet. Derfor er det ikke overraskende at det også finnes en variant av Gauss lov for det magnetiske feltet. Grunnet magnetfeltets retning kan det imidlertid ikke forklares av en potensialfunksjon. Det forklares i stedet av et vektorpotensiale $\mathbf{A}$ slik at $\mathbf{B}=\nabla \times \mathbf{A}$. Det betyr at divergensen av det magnetiske feltet er null. Altså
$$
\nabla \cdot \mathbf{B} = 0
$$
eller, om du vil,
$$
\iint_{\partial V} \mathbf{B} \cdot d\mathbf{A} = 0.
$$
Dette betyr at den størrelsen som spiller rollen som magnetfeltets masse, eller ladning, er konstant null. Av den grunn lar denne loven seg best formulere med ord, nemlig: {\it Det finnes ikke magnetiske monopoler}.


  Legg merke til at magnetfeltet allerede ser litt merkelig ut. Ut av ethvert volum vil det "sprute" like mye magnetfelt ut, som det "spruter" inn. Ytteligere, og mer seriøse, komplikasjoner ved magnetfeltet vil dukke opp når vi senere studerer kraften magnetfeltet utøver.




\subsection{Faradays induksjonslov}
På begynnelsen av 1830-tallet oppdaget fysikerene Michael Faraday og Joseph Henry et forhold mellom elektrisitet og magnetisme uavhengig. De oppdaget at spenningen over en krets er proporsjonal med endringen i magnetfeltets fluks ut av arealet kretsen omslutter.
Da Gustav Kirchhoff formulerte sin spenningslov, som sier at summen av spenningen over en strømsløyfe er alltid er lik null, i 1847 visste han altså at dette ikke alltid stemmer. I nærvær av et magnetisk felt i endring sier nemlig loven at summen av spenningen over en strømsløyfe er gitt ved hvor mye magnetfeltfluksen gjennom arealet utspent av sløyfen {\it avtar} med.


  La oss forsøke å skrive dette matematisk. Vi har sett at spenning, $V$, er det elektriske feltets potensialfunksjon, som betyr at
  $$
  \int_a^b \mathbf{E} \cdot d\mathbf{\ell} = V(b)-V(a)
  $$
  der $d\mathbf{\ell}$ er en differensial vektor som peker langs kurven somforbinder punktet $a$ med punktet $b$. Hvis kurven er lukket følger det altså at
  $$
  \oint_\gamma \mathbf{E} \cdot d\mathbf{\ell} = 0.
  $$
  Dette er Kirchhoffs spenningslov. Faradays induksjonslov sier at dette bare er et spesialtilfelle av den mer generelle loven
  $$
  \oint_\gamma \mathbf{E} \cdot d\mathbf{\ell} = -\frac{d}{dt}\iint_S \mathbf{B} \cdot d\mathbf{A}
  $$
  der $S$ er ethvert areal med rand lik $\gamma$. Dette er {\it Maxwell-Faradays lov} på integralform.



  Ved å benytte Stokes sats kan vi skrive om integralet langs sløyfen $\gamma$ som et integral over flaten $S$:
  $$
  \oint_\gamma \mathbf{E} \cdot d\ell = \iint_S \nabla \times \mathbf{E} \cdot d\mathbf{A} = -\frac{d}{dt}\iint_S \mathbf{B} \cdot d\mathbf{A}.
  $$
  Vi kan like gjerne utføre derivasjonen med hensyn på tid {\it før} integrasjonen, men da må vi passe på ikke å derivere noen av de variablene som integreres bort. Dette betyr at derivasjonen kan omgjøres til en partiell derivasjon inne i integralet slik at loven sier
  $$
  \iint_S \left( \nabla \times \mathbf{E} + \frac{\partial \mathbf{B}}{\partial t} \right)\cdot d\mathbf{A} = 0,
  $$
  der de to uttrykkene har blitt samlet under et felles integraltegn. Siden dette skal gjelde for {\it alle} flater $S$ følger det at
  $$
  \nabla \times \mathbf{E} =- \frac{\partial \mathbf{B}}{\partial t},
  $$
  som er Faradays induksjonslov på differensialform.




\subsection{Ampèré-Maxwells lov}
Som du kanskje allerede har gjettet bar den originale loven bare navnet til én av fysikerene, nemlig Ampèré. Ampèré oppdaget på begynnelsen av 1800-tallet at integralet rundt en lukket sløyfe er proporsjonal med strømmen gjennom sløyfen. Altså
$$
\oint_\gamma \mathbf{B} \cdot d\mathbf{\ell} = \mu_0 I_{enc}
$$
der $I$ er strømmen gjennom sløyfen $\gamma$ og $\mu_0$ proporsjonalitetskonstanten, kalt {\it vakuumpermittiviteten}. Strøm, som er definert som ladningsendring per tid, kan tenkes på som fluks av strømtetthet $\mathbf{J} \propto \mathbf{E}$. Altså
$$
I = \iint_S \mathbf{J}\cdot d\mathbf{A}.
$$
Uttrykt ved strømtettheten sier Ampèrés lov dermed at
$$
\oint_\gamma \mathbf{B} \cdot d\mathbf{\ell} = \mu_0 \iint_S \mathbf{J}\cdot d\mathbf{A}.
$$
Her Maxwell kommer inn i bildet. Maxwell oppdaget nemlig at dette ikke stemmer i nærheten av et elektrisk felt i endring. I det tilfellet må loven ta den mer generaliserte formen
$$
\oint_\gamma \mathbf{B} \cdot d\mathbf{\ell}
=
\mu_0 \iint_S
\mathbf{J}\cdot d\mathbf{A}
+
\mu_0 \varepsilon_0 \frac{d}{dt}\iint_S
\mathbf{E}\cdot d\mathbf{A}
=
\mu_0 \iint_S \left(
\mathbf{J}
+\varepsilon_0 \frac{\partial \mathbf{E}}{\partial t}
\right)\cdot d\mathbf{A}.
$$
I likhet med Faradays induksjonslov kan denne likningen skrives på formen
$$
\iint_S \left(\nabla \times \mathbf{B}
-\mu_0\mathbf{J}
-\mu_0\varepsilon_0 \frac{\partial \mathbf{E}}{\partial t}
\right)\cdot d\mathbf{A} = 0
$$
som, ettersom dette må gjelde for alle overflater $S$, betyr at
$$
\nabla \times \mathbf{B}
=
\mu_0\mathbf{J} + \mu_0\varepsilon_0 \frac{\partial \mathbf{E}}{\partial t}.
$$
Dette er Ampèré-Maxwells lov på differensialform.


  Loven sier at magnetfeltet rundt en lukket sløyfe er proporsjonal med summen av strømmen og endringen i den magnetiske fluksen gjennom sløyfen. Legg merke til likhetstrekket med Faradays induksjonslov. Hvis strømtettheten er null, er begge feltenes vridning proporsjonal med endringen i det andre feltet. En konsekvens av denne likningen er Biot-Savarts lov,
  $$
  \mathbf{B}(\mathbf{r}) = \frac{\mu}{2\pi}\int_\gamma \frac{I d\mathbf{\ell} \times \mathbf{r}'}{|\mathbf{r}'|^3}
  $$
  der I er strømmen langs en lukket sløyfe $\gamma$ hvorav $d\mathbf{\ell}$ er en infinitesimal vektor som peker langs sløyfen, mens $\mathbf{r}'=\mathbf{r}-\mathbf{\ell}$. Loven følger ved å anta stasjonært elektrisk felt og bruke at $\mathbf{B}=-\nabla^2 \mathbf{A}$ stemmer for $\nabla \cdot \mathbf{A}$. Selv om beviset utelates her er resultatet såpass viktig at det fortjener en plass i denne teksten: {\it Enhver elektrisk strøm gir opphav til et magnetfelt som står vinkelrett på strømretningen}.



\section{Lys som elektromagnetiske bølger}
La oss si at vi, helt umotivert, skulle ønske å undersøke uttrykket $\nabla \times (\nabla \times \mathbf{E})$. Ved å huske at
$$
\mathbf{a}\times (\mathbf{b} \times \mathbf{c}) = \mathbf{b}(\mathbf{a}\cdot \mathbf{c})-\mathbf{c}(\mathbf{a} \cdot \mathbf{b})
$$
ville vi først kunne observere at
$$
\nabla \times (\nabla \times \mathbf{E}) = \nabla (\nabla \cdot \mathbf{E}) - \nabla^2 \mathbf{E} = \nabla \frac{\rho}{\varepsilon_0} - \nabla^2 \mathbf{E}.
$$
I et område med jevnt fordelt ladningstetthet ($\nabla \rho = 0$) vil altså
$$
\nabla \times (\nabla \times \mathbf{E}) = - \nabla^2 \mathbf{E}.
$$
På den annen side har vi at
$$
\begin{aligned}
\nabla \times (\nabla \times \mathbf{E})
=
- \nabla \times \frac{\partial \mathbf{B}}{\partial t}
=
- \frac{\partial}{\partial t} \nabla \times \mathbf{B}
=
- \frac{\partial}{\partial t}
\left(
\mu_0 \mathbf{J}
+
\mu_0 \varepsilon_0 \frac{\partial \mathbf{E}}{\partial t}
\right)
\end{aligned}
$$
Hvis også strømtettheten i området er null, $\mathbf{J}=0$, vil altså det elektriske feltet tilfredsstille likningen
$$
\nabla^2 \mathbf{E}
=
\mu_0 \varepsilon_0 \frac{\partial^2 \mathbf{E}}{\partial t^2}.
$$
Dette likner veldig på den kjente partielle differensiallikningen
$$
\nabla^2 f
=
\frac{1}{c^2} \frac{\partial^2 f}{\partial t^2}
$$
kalt {\it bølgelikningen}. Bølgelikningen beskriver utviklingen til en bølge med hastighet $c$ i rom og tid. Vi har med andre ord funnet at det elektriske feltet oppfører seg som en bølge med hastighet
$$
c = \frac{1}{\sqrt{\mu_0 \varepsilon_0}}
$$
i områder uten strømtetthet og med jevnt fordelt ledningstetthet. Verdien av hastigheten er lyshastigheten:
$$
c = \frac{1}{\sqrt{\mu_0 \varepsilon_0}} \approx 3 \times 10^8 m/s.
$$
Dette er første steg i erkjennelsen om at lys er elektromagnetiske bølger. Legg merke til at siden det ikke er noen annen romlig avhengighet enn lyshastgheten i bølgelikningen for $\mathbf{E}$ kan dette også ses på som første skritt mot relativitetsteorien. For oss, som vet at lyshastigheten er den samme i alle referansesystemer, er det mulig å se at dette er tilfellet fra denne likningen -- det er faktisk en konsekvens av Maxwells lover!

Legg også merke til at lyshastigheten endres dersom permittiviteten eller permeabiliteten endrer seg. Lyshastigheten, $c'$, i et medium med permittivitet $\varepsilon$ og permeabilitet $\mu$ er med andre ord gitt ved
$$
	c' = \frac{1}{\sqrt{\varepsilon\mu}}.
$$
Endelig kan vi også gi litt mening til begrepet <i>brytningsindeks</i>. Siden brytningsindeksen $n$ til et medium er forholdet $c/c'$ mellom lyshastigheten $c$ i vakuum lyshastigheten $c'$ i mediet kan vi nemlig uttrykke brytningsindeksen ved
$$
n = \sqrt{\frac{\varepsilon\mu}{\varepsilon_0\mu_0}}.
$$

\section{Polarisering}
En pussig konsekvens av at lys er elektromagnetiske bølger er at lys er en vektorstørrelse. Med andre ord har lys en skjult retning -- retningen til det oscillerende $\mathbf{E}$-feltet! Parameteren som bestemmer denne retningen er gitt navnet <i>polarisering</i>. Vanlig, upolarisert lys, består av bølger med elektrisk komponent i alle retninger. I enkelte tilfeller hender det imidlertid at lyset blir polarisert i den forstand at $\mathbf{E}$-feltet til alt lyset har samme retning i hvert punkt. Et typisk eksempel er fenomenet <i>Brewstervinkel</i>. Ved en bestemt innfallsvinkel, $\theta_B=\arctan (n_{fra}/n_{til})$, går den komponenten av $\mathbf{E}$-feltet som ikke er parallell med overflaten rett inn i et medie uten å bli reflektert. Det er denne effekten man utnytter når man bruker polariserte solbrilleglass. Sollyset som reflekteres fra snøen er lineært polarisert. Dersom solbrilleglassene stopper lys med denne polariseringen vil refleksjonene fra snøen ikke være like sterke.




\section{Magnetisk kraft}
Kraften $\mathbf{F}$ det elektriske feltet $\mathbf{E}$ utøver på et legeme med ladning $q$ er gitt ved $\mathbf{F}=q \mathbf{E}$. Dette minner om Newtons andre lov, der $q$ spiller rollen som masse og $\mathbf{E}$ spiller rollen som legemets akselerasjon.
Den tilsvarende kraften fra et magnetfelt $\mathbf{B}$ er fundamentalt annerledes fra $\mathbf{E}$-feltets. Først og fremst er kraften fra et magnetfelt vinkelrett på magnetfeltets retning. Desto mer spesielt er det at kraften avhenger av det påvirkede legemets hastighet. Uttrykket for kraften er $\mathbf{F}=q\mathbf{v}\times \mathbf{B}$.
Dette er en katastrofe! Kraften på legemet avhenger av legemets hastighet. Sett fra legemets perspektiv er det altså ingen magnetkraft som virker på det. Dette høres svært inkonsistent ut.


  La oss se nærmere på et eksempel og prøve å se hva som foregår. Gitt en punktladning $q$ som beveger seg med hastighet $v$ parallellt med en uendelig lang og rett strømførende ledning med strøm $I$ som ligger i en avstand $r$ fra punktpartikkelen. La oss si at strømmen i ledningen er slik at de positive ladningene har samme hastighet $v$ som punktpartikkelen, mens de negative ladningene beveger seg like fort den andre veien. Strømmen kan da skrives $I=2\lambda v$ der $\lambda$ er ladningstettheten (ladning per lengde) for de positive ladningene. Siden de negative og positive ladningene i ledningen har samme ladningstetthet, men med forskjellig fortegn, er det altså ikke noe elektrisk felt som virker på punktpartikkelen. Punktladningen er imidlertid utsatt for en magnetisk kraft $F=-qvB$ der $B=\mu_0I/2\pi r$. Sett fra punktladningens ståsted utgjøres strømmen kun av de elektriske ladningenes bevegelse. Siden punktladningen, sett fra sitt eget perspektiv, ikke beveger seg kan det imidlertid ikke virke noen magnetisk kraft på punktladningen. Dette ser ut til å være et seriøst problem!


  Problemet løses av relativitetsteori. Når vi endrer referansesystem må huske at lengder sammentrekkes. At avstanden mellom de negative ledningene sammentrekkes mer enn de positive gjør at ledningen har en negativ ladning sett fra punktladningen. Denne ladningen setter opp et elektrisk felt som utøver en elektrisk kraft på den stillestående punktladningen. Med andre ord er magnetisme bare elektrisitet sett fra et referansesystem som beveger seg.













\newpage



\part*{Formler}

\begin{multicols}{2}
Maxwells lover
\begin{equation*}
\begin{aligned}
\nabla \cdot \mathbf{E} &= \frac{\rho}{\varepsilon_0} \ \ \ \ \ \ \ \ \ \ \ \ \text{(Gauss)} \\
\nabla \cdot \mathbf{B} &= 0 \ \ \ \ \ \ \ \ \ \ \ \ \text{(Magnetiske monopoler)}  \\
\nabla \times \mathbf{E} &= -\frac{\partial \mathbf{B}}{\partial t}  \ \ \ \ \ \ \ \text{(Faraday)} \\
\nabla \times \mathbf{B} &= \mu_0 \mathbf{J} + \frac{1}{c^2}\frac{\partial \mathbf{E}}{\partial t} \ \ \  \text{(Ampèré-Maxwell)}   \\
\end{aligned}
\end{equation*}

Andre formler:
\begin{equation*}
\begin{aligned}
\mathbf{F} &= q\left(\mathbf{E}+\mathbf{v}\times \mathbf{B}\right) \ \ \  \text{(Lorentz)}  \\
\mathbf{B} &= \frac{\mu}{2\pi}\int_\gamma \frac{I d\mathbf{\ell} \times \mathbf{r}'}{|\mathbf{r}'|^3} \ \ \  \text{(Biot-Savart)}  \\
\nabla^2 f &= \frac{1}{c^2} \frac{\partial^2 f}{\partial t^2} \ \ \ \ \ \ \ \ \ \text{(Bølgelikningen)}
\end{aligned}
\end{equation*}

\end{multicols}



\begin{comment}
	$$
  L_+' = \frac{ L_+ }{ \sqrt{1-\frac{0^2}{c^2}} } = L_+
  $$
  sett fra punktpartikkelen. På samme måte er avstanden mellom de negative ladningene, som for oss er $L_-$, gitt ved
  $$
  L_-' = L_- \sqrt{1-\frac{v^2}{c^2}}.
  $$
  Det betyr at ladningstettheten i ledningen sett fra punktpartikkelen er
  $$
  \lambda' = \lambda\sqrt{1-\frac{v^2}{c^2}}-\frac{\lambda}{\sqrt{1-\frac{v^2}{c^2}}}
  $$
  som hvis $v$ er mye mindre enn $c$ gir
  $$
  \lambda' = -\lambda\frac{v^2}{c^2}.
  $$
  Dette oppretter et elektrisk felt
  $$
  E = \frac{1}{\varepsilon_0}\frac{\lambda'}{2\pi r} = -\frac{1}{2\pi r}\frac{\lambda v^2}{\varepsilon_0c^2}
  $$
  som gir opphav til en kraft
  $$
  F = -q\frac{1}{2\pi r}\frac{\lambda v^2}{\varepsilon_0c^2}
  $$
\end{comment}











\part*{Oppgaver}
\subsection{Gauss lov}
La

\begin{equation}
\mathbf{E}(r) =
\begin{cases}
\frac{Qr}{4\pi \epsilon_0 R^3} \mathbf{e}_r &\text{ for } r \leq R \\
\frac{Q}{4\pi \epsilon_0 r^2} \mathbf{e}_r  &\text{ for } r > R
\end{cases}
\label{eq:Esphere}\end{equation}

være det elektriske feltet fra ladning $Q$, hvor $r$ er avstanden fra origo, $\epsilon_0$ er vakuumpermittiviteten og $R$ er et eller annet positivt tall. Gradienten i kulekoordinater er:
\begin{equation}
\nabla =
 \mathbf{e}_r \frac{1}{r^2}\frac{\partial}{\partial r}r^2
+\mathbf{e}_\theta \frac{1}{r\sin \theta}\frac{\partial}{\partial\theta}\sin \theta
+\mathbf{e}_\phi \frac{1}{r\sin \theta}\frac{\partial}{\partial\phi}.
\label{eq:sphereNabla}\end{equation}

\begin{description}
	\item[a)] Hva er ladningstettheten $\rho(r)$?
		[Hint: {\it Se på de to tilfellene $r\leq R$ og $r>R$ hver for seg. }]
	\item[b)] Hvilken form har det ladede legemet?
	\item[b)]
	Man kan også gå den andre veien ved å finne det elektriske feltet $\mathbf{E}$ for et ladet legeme av en bestemt form. Hva må vi anta om ladningstettheten inne i figuren for å finne tilbake til uttrykket i likning (\ref{eq:Esphere})?
\end{description}



\subsection{Ampèrés lov}
Bruk Ampèrés lov til å finne det magnetiske feltet $\mathbf{B}$ i en avstand $r$ fra en uendelig lang, rett ledning med strøm $I$.



\subsection{Faradays lov}
Vi legger en strømførende sløyfe langs randen av en sirkel $S$ slik at den lukkes med én vinding.
Sløyfen ligger innenfor et justerbart homogent magnetfelt som peker parallelt med normalvektoren til $S$.
Vi endrer magnetfeltet slik at feltstyrken er gitt av $B(t)=B_0 \cos (\omega t)$.
\begin{description}
	\item[a)] Finn et uttrykk for det elektriske feltet $\mathbf{E}$ som induseres langs strømsløyfen.
		[Hint: {\it Faradays induksjonslov på integralform }]
	\item[b)] Finn et uttrykk for den induserte spenningen i strømsløyfen.
	\item[c)] Hvordan vil spenningen endre seg dersom vi øker vinkelfrekvensen $\omega$?
	\item[d)] Hva skjer dersom strømsløyfens radius blir større? La $r\rightarrow \infty$, er svaret realistisk?
	\item[e)] Hva endrer seg i uttrykket for spenningen dersom den strømførende sløyfen i stedet lukkes med $N$ vindinger?
	\item[f)] Hva endrer seg i uttrykket for det elektriske feltet dersom den strømførende sløyfen i stedet lukkes med $N$ vindinger?
\end{description}




\subsection{Elektromagnetiske bølger}
\begin{description}
	\item[a)] Anta at $\mathbf{J}=0$ og $\rho=0$ og bruk Maxwells lover til å vise at også $\mathbf{B}$-feltet tilfredsstiller en bølgelikning.
	\item[b)] Hvordan ser likningen ut dersom vi ikke antar noe om $\mathbf{J}$ og $\rho$?
	\item[c)] Hvilken differensiallikning må $\mathbf{J}$ og $\rho$ tilfredsstille for at det magnetiske feltet skal oppføre seg som en bølge?
		[Hint: {\it Hva må være null for at uttrykket fra b) blir en bølgelikning? }]
	\item[d)] Hva er vinkelen mellom $\mathbf{E}$ og $\mathbf{B}$ dersom magnetfeltets retning ikke endrer seg over tid? At magnetfeltets retning ikke endrer seg over tid betyr at $\partial \mathbf{B}/\partial t = \hat{\mathbf{n}} \partial B/\partial t$ der $\hat{\mathbf{n}}$ er magnetfeltets retning.
		[Hint: {\it Bruk Faradays induksjonslov på differensialform. }]
\end{description}



\subsection{Polarisering}
\begin{description}
	\item[a)] Vis at $\mathbf{E}(r,x)=E_0\cos(kz-\omega t) \mathbf{e}_x$ tilfredstiller bølgelikningen. Hva må $c$, $k$ og $\omega$ i såfall tilfredsstille?
	\item[b)] I hvilken retning beveger bølgen seg?
	\item[c)] I hvilken retning peker $\mathbf{E}$-feltet?
	\item[a)] Vis at $\mathbf{E}(r,x)=E_0\cos(kz-\omega t) \mathbf{e}_x+E_0\sin(kz-\omega t) \mathbf{e}_y$ tilfredstiller bølgelikningen.
	\item[c)] I hvilken retning peker $\mathbf{E}$-feltet nå?
\end{description}


\subsection{Ladningstetthet}
\begin{description}
	\item[a)] Bruk Ampèré-Maxwells lov til å uttrykke divergensen til $\nabla \times \mathbf{B}$ ved $\mathbf{J}$ og $\mathbf{E}$.
	\item[b)] Bruk at divergensen til en virvling alltid er null til å bli kvitt $\mathbf{B}$ fra uttrykket for $\mathbf{J}$ og $\mathbf{E}$.
	\item[c)] Bruk Gauss lov til å finne en likning som relaterer strømtettheten $\mathbf{J}$ med ladningstettheten $\rho$.
	\item[d)] Bruk at $\mathbf{J}=\rho \mathbf{v}$, der $\rho$ er ladningstetthet og $\mathbf{v}$ er ladningens gjennomsnittlige drivhastighet, til å skrive om likningen slik at den bare avhenger av $\rho$ og $\mathbf{v}$. Kjenner du igjen likningen? Hva betyr dette for mengden ladning i universet?
		[Hint: {\it Kontinuitetslikningen. }]
\end{description}




\end{document}
