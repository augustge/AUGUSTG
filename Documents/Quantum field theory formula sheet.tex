


\documentclass[a4paper, norsk, 8pt, landscape]{article}
\usepackage[utf8]{inputenc}
\usepackage[T1]{fontenc}
\usepackage{babel, textcomp, color, amsmath, amssymb, tikz, subfig, float,esint}
\usepackage{amsfonts}
\usepackage{graphicx}
\usepackage{multicol}
\usepackage{slashed}
\usepackage{tikz}
\usepackage{pgfplots}

\newcommand{\EQU}[1] { \begin{equation*} \begin{split}
#1
\end{split} \end{equation*} }
 \newcommand{\DE}[1] {  \begin{description}  #1 \end{description} }
 \newcommand{\IT}[2] { \item[\color{blue} #1]{#2} }
 \newcommand{\vv}[1] { \mathbf{#1} }
 \newcommand{\PAR}[2]{ \frac{\partial #1}{\partial #2}}
 \newcommand{\expe}[1] { \left\langle#1\right\rangle }
 \newcommand{\ket}[1] { |#1\rangle }
  \newcommand{\bra}[1] { \langle #1 | }
  \newcommand{\braket}[2] { \langle #1 | #2 \rangle }
  \newcommand{\commutator}[2]{ \left[ #1 , #2\right] }
  \newcommand{\colvec}[2] {
  \left( \begin{matrix}
 #1 \\
 #2 \\
  \end{matrix}\right) }
 \newcommand{\PLOTS}[4]{
\begin{tikzpicture}
\begin{axis}[
    axis lines = #3, %usally left
    xlabel = #1,
    ylabel = #2,
]
#4
\end{axis}
\end{tikzpicture}
}


\newcommand{\addPLOT}[4]{
\addplot [domain=#1:#2,samples=200,color=#3,]{#4};}
\newcommand{\addCOORDS}[1]{\addplot coordinates {#1};}
\newcommand{\addDRAW}[1]{\draw #1;}
\newcommand{\addNODE}[2]{ \node at (#1) {#2};}

%		\PLOTS{x}{y}{left}{
%			\ADDPLOT{x^2}{-2}{2}{blue}
%			\ADDCOORDS{(0,1)(1,1)(1,2)}
%		}




\definecolor{svar}{RGB}{0,0,0}
\definecolor{opgavetekst}{RGB}{109,109,109}
\definecolor{blygraa}{RGB}{44,52,59}

\hoffset = -60pt
\voffset = -95pt
\oddsidemargin = 0pt
\topmargin = 0pt
\textheight = 0.97\paperheight
\textwidth = 0.97\paperwidth

\begin{document}
% \tiny
\footnotesize
\begin{multicols*}{3}

\subsection*{\footnotesize  DIMENSIONAL ANALYSIS}
Natural units: $c=\hbar=1 \implies L=t=E^{-1}=m^{-1}$. \\
Fields [units of energy]: $[\phi] \sim [A_\mu^a] \sim 1$ and $[\psi] \sim 3/2$. \\
Cross section: $\sigma \sim -2$.















\subsection*{\footnotesize QUANTUM MECHANICS}
$E=i\partial_t$, $p_j=-i\partial_j$, combined into $p^\mu = i \partial^\mu$. \\
 {\textbf{Pauli matrices:}} \\
$
\sigma^1 =
\left(\begin{matrix}
0 & 1 \\
1 & 0 \\
\end{matrix}\right),
\sigma^2 =
\left(\begin{matrix}
0 & -i \\
i & 0 \\
\end{matrix}\right) \text{ and }
\sigma^1 =
\left(\begin{matrix}
1 & 0 \\
0 & -1 \\
\end{matrix}\right)
$ \\
satisfying $\sigma^i \sigma^j= \delta^{ij}+i\varepsilon^{ijk}\sigma^k$.
\\


















\subsection*{\footnotesize MATHEMATICAL}
 {\textbf{Dirac delta:}} \\
$\int d^4 x e^{ikx} = (2\pi)^4\delta^{(4)}(k)$, \ \
$\delta(f(x)-f(x_0))=\frac{\delta(x-x_0)}{|f'(x_0)|}$ \\
If $g(x)$ has real zeros $\{x_i\}$ then
\[\delta(g(x))=\sum_i \frac{\delta(x-x_i)}{|g'(x_i)|}\]
 {\textbf{Fourier transform:}} \\
$\mathcal{F}(f)=\int d^4xe^{ikx}f(x)$, \ \
$\mathcal{F}^{-1}(\hat{f})=\int \frac{d^4k}{(2\pi)^4}e^{-ikx}\hat{f}(k)$
 \\
 {\textbf{Euler-Lagrange:}}
The action $S=\int L dt$ is extremal when \\
\[
\partial_\mu \frac{\partial \mathcal{L}}{\partial (\partial_\mu \phi^a)}
- \frac{\partial \mathcal{L}}{\partial \phi^a }=0.
\]
Momentum density conjugate $\pi(\mathbf{x})=\partial \mathcal{L}/\partial \dot \phi(\mathbf{x})$.
Hamiltonian: $\mathcal{H}=\pi(\mathbf{x})\dot{\phi}(\mathbf{x})-\mathcal{L}$.

















\subsection*{\footnotesize DIRAC BILINEARS AND TRACE TECHNOLOGY}
 {\textbf{Gamma matrices:}}
Satisfies the Lorentz algebra. In the Weyl basis \\
$
\gamma^0 =
\left(\begin{matrix}
0 & \mathbb{I}_2 \\
\mathbb{I}_2 & 0 \\
\end{matrix}\right),
\gamma^k =
\left(\begin{matrix}
0 & \sigma^k \\
-\sigma^k & 0 \\
\end{matrix}\right) \text{ and }
\gamma^5 =
\left(\begin{matrix}
-\mathbb{I}_2 & 0\\
0 & \mathbb{I}_2 \\
\end{matrix}\right)
$ \\
satisfying $\{\gamma^\mu, \gamma^\nu \} = 2g^{\mu \nu} \mathbb{I}_4$, and
$(\gamma^\mu)^\dagger = \gamma^0 \gamma^\mu \gamma^0$.
\\
 {\textbf{Dirac bilinears:}} Since complex $4\times 4$ matrices has 32
degrees of freedom. Imposing $(\gamma^k)^\dagger = - \gamma^k$ and
$(\gamma^0)^\dagger=\gamma^0$ leaves 16 degrees of freedom. These are \\
\begin{tabular}{l l r}
  $\Gamma_i$                                        & $\overline{\psi}\Gamma_i \psi$  & Number\\    \hline
  $\mathbb{I}$                                      & scalar                          & 1    \\
  $\gamma^\mu$                                      & vector                          & 4    \\
  $S^{\mu\nu}=\frac{i}{2}[\gamma^\mu,\gamma^\nu]$   & tensor                          & 6    \\
  $\gamma^5$                                        & pseudo-scalar                   & 1    \\
  $\gamma^5\gamma^\mu$                              & pseudo-vector                   & 4    \\
\end{tabular} \\
Here we have defined $\gamma^5 = i \gamma^0 \gamma^1 \gamma^2 \gamma^4$. This satisfies: \\
$\{\gamma^5,\gamma^\mu\}=0$. Note also that $P_{R/L}=\frac{1\pm \gamma^5}{2}$ projects to lower two, or upper two
components respectively. With $\gamma^\mu P_{R/L} = P_{L/R} \gamma^\mu$
\\
 {\textbf{Contraction identities:}} \\
$\gamma^\mu \gamma_\mu = 4 \times \mathbb{I}$, \ \ \
$\gamma^\mu \gamma^\nu \gamma_\mu = -2 \gamma^\nu$, \ \ \
$\gamma^\mu \gamma^\nu \gamma^\rho \gamma_\mu = 4g^\rho \nu \mathbb{I}$, \ \ \
\\
 {\textbf{Useful ''slashed''-tricks:}} \\
$\slashed{p}\slashed{p}=p^2\mathbb{I}$, \ \ \
$\gamma^\mu \slashed{p} \gamma_\mu = -2\slashed{p}$, \ \ \
$\{\slashed{A},\slashed{B}\}=\slashed{A}\slashed{B}+\slashed{B} \slashed{A}=2A \cdot B$, \ \ \
$\gamma^\mu \slashed{p} \gamma_\mu = -2\slashed{p}$
\\
 {\textbf{Trace technology:}} \\
$\text{tr}(A+B)=\text{tr}(A)+\text{tr}(B)$ \\
$\text{tr}(cA)=c\text{tr}(A)$ \\
$\text{tr}(\gamma^\mu \gamma^\nu) = 4g^{\mu\nu}$ \\
$\text{tr}(\gamma^\mu \gamma^\nu \gamma^\rho \gamma^\sigma)
=
4(g^{\mu\nu}g^{\rho\sigma}-g^{\mu\rho}g^{\nu\sigma}+g^{\mu\sigma}g^{\nu\rho})$ \\
$\text{tr}(\text{any odd \# of }\gamma\text{-matrices})=0$ \\
$\text{tr}(\gamma^5)=0$ \\
$\text{tr}(\slashed{a}\slashed{b})=4(a\cdot b)$














\subsection*{\footnotesize  WEYL SPINORS}
Set $\psi_R = P_R \psi$ and $\psi_L= P_L \psi$ and write the Dirac equation as
\[
\left(\begin{matrix}
-m                                  &   i\sigma^\mu \partial_\mu \\
i\overline{\sigma}^\mu \partial_\mu &   -m \\
\end{matrix}\right)
\left(\begin{matrix}
\psi_L \\
\psi_R \\
\end{matrix}\right)=0
\]
where $\sigma^\mu \equiv (1,\sigma^j)$ and $\overline{\sigma}^\mu \equiv (1,-\sigma^j)$. Note that for $m=0$
this separates into two independent equations.













\subsection*{\footnotesize  PLANE WAVE SOLUTIONS (Dirac eq)}
The plane wave solutions are $\psi(x)=u(p)e^{-ipx}$ for positive frequency (particles) and
$\psi(x)=v(p)e^{ipx}$ for negative frequency (anti-particles). Here we have
\EQU{
u^s(p) &=
\left(\begin{matrix}
\sqrt{p_\mu \sigma^\mu }\xi^s \\
\sqrt{p_\mu \overline{\sigma}^\mu }\xi^s \\
\end{matrix}\right)
=
\frac{1}{\sqrt{2(m+E_\mathbf{p})}}
\left(\begin{matrix}
[E+m-\mathbf{p}\cdot \vec{\sigma}]\xi^s \\
[E+m+\mathbf{p}\cdot \vec{\sigma}]\xi^s \\
\end{matrix}\right) \\
v^s(p) &=
\left(\begin{matrix}
\sqrt{p_\mu \sigma^\mu }\eta^s \\
-\sqrt{p_\mu \overline{\sigma}^\mu }\eta^s \\
\end{matrix}\right)
=
\frac{1}{\sqrt{2(m+E_\mathbf{p})}}
\left(\begin{matrix}
[E+m-\mathbf{p}\cdot \vec{\sigma}]\xi^s \\
-[E+m+\mathbf{p}\cdot \vec{\sigma}]\xi^s \\
\end{matrix}\right). \\
}
These satisfy
$\overline{u}^r(p)u^s(p)=2m\delta^{rs}$, \
$\overline{v}^r(p)v^s(p)=-2m\delta^{rs}$, \
$\overline{v}^r(p)u^s(p)=\overline{u}^r(p)v^s(p)=0$, \
$u^{r\dagger}(p)u^s(p)=2E_\mathbf{p}\delta^{rs}$, \
$v^{r\dagger}(p)v^s(p)=2E_\mathbf{p}\delta^{rs}$.
Perhaps more important is
\[
\sum_{s=1,2} u^s(p) \overline{u}^s(p) = \slashed{p}+m
\text{ and }
\sum_{s=1,2} v^s(p) \overline{v}^s(p) = \slashed{p}-m
\]












\subsection*{\footnotesize  COMMUTATION RELATIONS}
Ladder operators for bosons:
$[a_\mathbf{p},a_\mathbf{k}^\dagger]=(2\pi)^3 \delta^{(3)}(\mathbf{p}-\mathbf{k})$.
\\
Bosonic fields:
$[\phi(\mathbf{x}),\pi(\mathbf{y})]=i\delta^{(3)}(\mathbf{x}-\mathbf{y})$ with
$[\phi(\mathbf{x}),\phi(\mathbf{y})]
=[\pi(\mathbf{x}),\pi(\mathbf{y})]=0$. \\
Bosonic hamiltonian:
$[H,a^\dagger_\mathbf{p}] = E_\mathbf{p}a_\mathbf{p}^\dagger$
and
$[H,a_\mathbf{p}] = -E_\mathbf{p}a_\mathbf{p}^\dagger$.













\subsection*{\footnotesize  TRANSFORMATIONS}
\textbf{Lorentz group:} \\
$ x^\mu \mapsto \Lambda^\mu_{\ \nu} x^\nu$,\\
$\phi(x) \mapsto \phi(\Lambda^{-1}x)$, \\
$\partial_\mu \phi(x) \mapsto (\Lambda^{-1})_\mu^{\ \nu}(\partial_\nu \phi)(\Lambda^{-1}x)$.
\\
\textbf{Spinor transformations:} \\
$\Lambda_{1/2}=\exp\left(-\frac{i}{2}\omega_{\mu\nu}S^{\mu\nu}\right)$, \\
$\Lambda^{-1}_{1/2}\gamma^\mu \Lambda_{1/2} = \Lambda^\mu_{\ \nu} \gamma^\nu$
\\
\textbf{Gauge transformations:} \\
$\psi \mapsto V\psi = e^{i\alpha^a t^a}\psi \simeq (1+i\alpha^a t^a) \psi$, \\
$A_\mu^a t^a \mapsto V(x) \left( A_\mu^a(x)t^a + \frac{i}{g}\partial_\mu \right)V^\dagger(x)$ (finite) \\
$A_\mu^a \mapsto A_\mu^a + \frac{1}{g}\partial_\mu  \alpha^a +f^{abc}A_\mu^b \alpha^c$ (infinitesimal) \\
Set $t^a = 1$ and $f^{abc}=0$ for abelian. Possibly also $g \mapsto -e$.







\subsection*{\footnotesize  SYMMETRIES}
 {\textbf{Noethers theorem:}}
Under an infinitesimal transformation $\phi \mapsto \phi'=\phi+\alpha \Delta \phi$ the
Lagrangian transforms with $\mathcal{L}\mapsto \mathcal{L}'=\mathcal{L}+\alpha \partial_\mu \mathcal{J}^\mu$.
Then $\partial_\mu j^\mu=0$ for conserved current $j^\mu(x)=\frac{\partial\mathcal{L}}{\partial(\partial_\mu\phi)} \Delta \phi - \mathcal{J}^\mu$.
Conserved charge $Q=\int j^0 d^3 x$.
\\
 {\textbf{CONTINUOUS SYMMETRIES:}} \\
 {\textbf{Dirac equation:}}
$\psi \mapsto e^{i\alpha}\psi$ gives conservation of $j^\mu = \overline{\psi}\gamma^\mu\psi$. \\
$\psi \mapsto e^{i\alpha\gamma^5}\psi$ is symmetry when $m=0$. Then $\overline{\psi}\gamma^\mu \gamma^5 \psi$ is conserved. \\
$x^\mu \mapsto x^\mu + a^\mu$ gives $\phi(x) \mapsto a^\mu \partial_\mu \phi(x)$. Four conserved currents:
$T_\nu^\mu = \frac{\partial \mathcal{L}}{\partial (\partial_\mu \phi)} \partial_\nu \phi - \mathcal{L}\delta_\nu^\mu$.
Conserved charges are $H = \int d^3 x T^{00} $ (energy) and $P^i = \int d^3 x T^{oi}$ (physical momentum).
\\
 {\textbf{DISCRETE SYMMETRIES:}} \\
 {\textbf{C (charge conjugation):}} $C\psiC=-i(\overline{\psi}\gamma^0 \gamma^2)^T$ \\
Constructed from $a_\mathbf{p}^s \mapsto b_{\mathbf{p}}^s$ and
$b_\mathbf{p}^s \mapsto a_{\mathbf{p}}^s$.
\\
 {\textbf{P (parity):}} $P\psi(t,\mathbf{x}) P = \eta_a \gamma^0 \psi(t,-\mathbf{x})$ \\
with $|\eta|^2=1$. Constructed from $a_\mathbf{p}^s \mapsto \eta_a a_{-\mathbf{p}}^s$ and
$b_\mathbf{p}^s \mapsto \eta_b b_{-\mathbf{p}}^s$. Parity is mirror symmetry.
\\
Examples of even:
$t$, $m$, $E$, $P$ (power), $\rho$, $V$, $\mathbf{L}$, $\mathbf{B}$, ... \\
Examples of odd:
$h$ (helicity), $\mathbf{x}$, $\mathbf{v}$, $\mathbf{p}$, $\mathbf{F}$, $\mathbf{E}$, $\mathbf{A}$, ...
\\
 {\textbf{T (time reversal):}} $T\psi(t,\mathbf{x})T=(-\gamma^1 \gamma^3)\psi(-t,\mathbf{x})$ \\
Constructed from $a_\mathbf{p}^s \mapsto a_{-\mathbf{p}}^{-s}$ and
$b_\mathbf{p}^s \mapsto \eta_b b_{-\mathbf{p}}^{-s}$ with \\$Tc=c^*T$ for $c\in\mathbb{C}$.
Then $Te^{iHt}=e^{-iHt}T$. \\
Examples of even:
$\mathbf{x}$, $\mathbf{a}$, $\mathbf{F}$, E, $\mathbf{E}$, $\rho$, ... \\
Examples of odd:
t, $\mathbf{v}$, $\mathbf{p}$, $\mathbf{L}$, $\mathbf{A}$, $\mathbf{B}$, $\mathbf{j}$, ...
\\
 {\textbf{Summary:}}
If $(-1)^\mu=1$ for $\mu=0$ and $(-1)^\mu=-1$ for $\mu=1,2,3$ then \\
\resizebox{\columnwidth}{!}{
\begin{tabular}{l | r r r r r r}
    & $\overline{\psi}\psi$ & $i\overline{\psi}\gamma^5\psi$  & $\overline{\psi}\gamma^\mu\psi$ & $\overline{\psi}\gamma^\mu \gamma^5 \psi$   & $\overline{\psi}S^{\mu\nu}\psi$     & $\partial_\mu$  \\  \hline
P   & $+1$                  & $-1$                            & $(-1)^\mu$                      & $-(-1)^\mu$                                 & $(-1)^\mu (-1)^\nu$                 & $(-1)^\mu$      \\
T   & $+1$                  & $-1$                            & $(-1)^\mu$                      & $(-1)^\mu$                                  & $-(-1)^\mu (-1)^\nu$                & $-(-1)^\mu$     \\
C   & $+1$                  & $+1$                            & $-1$                            & $+1$                                        & $-1$                                & $+1$            \\
CPT & $+1$                  & $+1$                            & $-1$                            & $-1$                                        & $+1$                                & $-1$            \\
\end{tabular}}

















\subsection*{\footnotesize  LAGRANGIANS}
 {\textbf{Free scalar field (Klein-Gordon):}}
$\mathcal{L}_{\text{KG}}=\frac{1}{2}(\partial_\mu \phi)^2-\frac{1}{2}m^2 \phi^2$.
\\
 {\textbf{$\phi^4$-Theory:}}
$\mathcal{L}=\mathcal{L}_{\text{KG}}-\frac{\lambda}{4!}\phi^4 =\frac{1}{2}(\partial_\mu \phi)^2-\frac{1}{2}m^2 \phi^2 -\frac{\lambda}{4!}\phi^4$.
\\
 {\textbf{Free Dirac theory:}}
$\mathcal{L}_{\text{Dirac}}=\overline{\psi}(i\slashed{\partial}-m)\psi$.
\\
 {\textbf{Maxwell theory (electromagnetism):}}
$\mathcal{L}_{\text{Maxwell}}=- \frac{1}{4}(F_{\mu\nu})^2$.
\\
 {\textbf{Yukawa theory:}}
$\mathcal{L}_{\text{Yukawa}}=\mathcal{L}_{\text{KG}}+\mathcal{L}_{\text{Dirac}}-g\overline{\psi}\psi \phi$.
\\
 {\textbf{QED:}}
$\mathcal{L}_{\text{QED}}=\overline{\psi}(i\slashed{\partial}-m)\psi
- \frac{1}{4}(F_{\mu\nu})^2-e\overline{\psi}\gamma^\mu \psi A_\mu$ \\
or
$\mathcal{L}_{\text{QED}}=\overline{\psi}(i\slashed{D}-m)\psi
- \frac{1}{4}(F_{\mu\nu})^2$. Here $D_\mu=\partial_\mu+ieA_\mu$.
\\
 {\textbf{Complex scalar QED:}}
$\mathcal{L}= -\frac{1}{4}(F_{\mu\nu})^2 +(D_\mu \phi)^* (D^\mu \phi) -m^2 \phi^* \phi$ \\
where $D_\mu = \partial_\mu + ie A_\mu$.
\\
 {\textbf{Yang-Mills for fermion multiplet:}}
$\mathcal{L}_{\text{YM}}= \overline{\psi}(i\slashed{D}-m)\psi-\frac{1}{4}(F_{\mu\nu}^a)^2$ \\
where $(D_\mu \psi)_a = \partial_\mu \phi_a + gf^{abc}A_\mu^b \psi_c
= \partial_\mu \phi_a - igA_\mu^a t^a \psi_c$. Written out: \\
$\mathcal{L}_{\text{YM}}=
\mathcal{L}_{\text{Dirac}}
-\frac{1}{4}(\partial_\mu A_\nu^a - \partial_\nu A_\mu^a)^2
+gA_\mu^a\overline{\psi}\gamma^\mu t^a \psi
-gf^{abc}(\partial_\mu A_\nu^a)A^{\mu b}A^{\nu c}
-\frac{1}{4}g^2(f^{eab}A_\mu^a A_\nu^b)(f^{ecd}A^{\mu c}A^{\nu d}).$
\\
 {\textbf{Faddeev-Popov (non-abelian ghosts):}} \\
$\mathcal{L}_{FP} = \mathcal{L}_{\text{YM}}
- \frac{1}{2\xi}\left(\partial^\mu A_\mu^a\right)^2
- \overline{c}(\partial^\mu D_\mu)c$. $c$ and $\overline{c}$ are grassmann
valued scalar fields (wrong spin-statistics). $\xi$ determines the specific gauge
(e.g. $\xi=1$ for Feynman gauge). Ghosts are adjoint.
\\
 {\textbf{Linear sigma model:}} \\
$\mathcal{L}_{LSM} = \frac{1}{2}(\partial_\mu \sigma^i)^2+\frac{1}{2}\mu^2(\phi^i)^2-\frac{\lambda}{4}[(\phi^i)^2]^2$
for $N$ real scalar fields $\phi^i$. Potential has minimum for $(\phi_0)^i=\frac{\mu^2}{\lambda}$.
Invariant under $\phi^i \mapsto R^{ij}\phi^j$ for orthogonal matrices $R$.
















\subsection*{\small  OPERATORS}
 {\textbf{Helicity:}}
$h=\hat{p}\cdot \mathbf{S}=\frac{1}{2}\hat{p}_i
\left(\begin{matrix}
\sigma^i &        0 \\
0        & \sigma^i \\
\end{matrix}\right)$
\\
 {\textbf{Dirac spinors:}} \\
$\psi(x)=\int \frac{d^3 p}{(2\pi)^3}
\frac{1}{\sqrt{2E_\mathbf{p}}}
\sum_s^2\left(a_\mathbf{p}^s u^s(p)e^{-ipx}+b_\mathbf{p}^{s\dagger}  v^s(p)e^{ipx}\right)$ \\
$\overline{\psi}(x)=\int \frac{d^3 p}{(2\pi)^3}
\frac{1}{\sqrt{2E_\mathbf{p}}}
\sum_s^2\left(b_\mathbf{p}^s \overline{v}^s(p)e^{-ipx}+a_\mathbf{p}^{s\dagger} \overline{u}^s(p)e^{ipx}\right)$
\\
 {\textbf{Real scalar field:}} \\
$\phi(x)=\int \frac{d^3 p}{(2\pi)^3} \frac{1}{\sqrt{2E_\mathbf{p}}}
\left(a_\mathbf{p}e^{-ipx}+a_\mathbf{p}^\dagger e^{ipx}\right)$
\\
 {\textbf{Electromagnetic gauge field:}} \\
$A_\mu(x)=\int \frac{d^3 p}{(2\pi)^3}
\frac{1}{\sqrt{2E_\mathbf{p}}}
\sum_r^3\left(a_\mathbf{p}^r \varepsilon^r(p)e^{-ipx}+(a_\mathbf{p}^r)^\dagger \varepsilon^{r*}(p) e^{ipx}\right)$

















\subsection*{\small  PROPAGATORS}
 {\textbf{Klein-Gordon propagator:}} \\
$D_F(x-y)
=\int \frac{d^4 p}{(2\pi)^4}\frac{ie^{-ip(x-y)}}{p^2-m^2+i\varepsilon}
=\bra{0}\mathbb{T} \phi(x)\phi(y)\ket{0}$ \\
is the Green's function for KGeq: \\
$(-\partial^2-m^2)D_F(x-y)=i\delta^{(4)}(x-y)$
\\
 {\textbf{Dirac propagator:}} \\
$S_F(x-y)
=\int \frac{d^4 p}{(2\pi)^4}\frac{i(\slashed{p}+m)e^{-ip(x-y)}}{p^2-m^2+i\varepsilon}
=\bra{0}\mathbb{T} \psi(x)\overline{\psi}(y)\ket{0}$ \\
is the Green's function of the Dirac operator: \\
$(i\slashed{\partial}-m)S_F(x-y)=i\delta^{(4)}(x-y)$

















\subsection*{\small  GROUP THEORY}
Associated to any Lie group $G$ is its Lie algebra $\mathfrak{G}$,
defined to be the tangent space of $G$ at the identity. Generally $[t^a,t^b]=if^{abc}t^{c}$ and \\
$f^{ade}f^{bcd}+f^{bde}f^{cad}+f^{cde}f^{abd}=0$ (Jacobi identity).
\begin{itemize}
  \item
    $SU(n)$ has $n^2-1$ independent matrices:\\
    $SU(n)=\{M\in GL(n,\mathbb{C})|MM^\dagger=\mathbb{I}\}$ \\
    $\mathfrak{su}(n)=\{M\in \mathfrak{gl}(n,\mathbb{C})|M+M^\dagger=0\}$
  \item
    $SO(n)$ has $n(n-1)/2$ independent matrices:\\
    $SO(n)=\{M\in GL(n,\mathbb{R})|MM^T=\mathbb{I}\}$ \\
    $\mathfrak{so}(n)=\{M\in \mathfrak{gl}(n,\mathbb{R})|M+M^T=0\}$
  \item
    $SP(n)$ has $n(n-1)/2$ independent matrices:\\
    $SP(n)=\{M\in GL(2n,\mathbb{R})|MJM^T=\mathbb{I}\}$ \\
    $\mathfrak{sp}(n)=\{M\in \mathfrak{gl}(2n,\mathbb{R})|MJ+JM^T=0\}$ \\
    for $J=\left(\begin{matrix}
    0 & \mathbb{I}_n \\
    -\mathbb{I}_n & 0 \\
    \end{matrix}\right)$
\end{itemize}
\\
 {\textbf{Lorentz algebra:}} \\
$\frac{1}{i}[J^{\mu\nu},J^{\rho \sigma}]=
g^{\nu\rho}J^{\mu\sigma}-g^{\mu\rho}J^{\nu\sigma}-g^{\nu\sigma}J^{\mu\rho}+g^{\mu\sigma}J^{\nu\rho}$ \\
Ordinary four-vectors:
$(\mathcal{J}^\mu\nu)_{\alpha \beta} = i(\delta^\mu_\alpha \delta^\nu_\beta-\delta^\nu_\alpha \delta^\mu_\beta)$. \\
General transformation:
$V^\alpha \mapsto \left(\delta^{\alpha}_\beta - \frac{i}{2}\omega_{\mu\nu}(\mathcal{J}^{\mu\nu})^\alpha_{\ \beta} \right)V^\beta$.
\\
 {\textbf{Dirac algebra:}} \\
$\{\gamma^\mu,\gamma^\nu\}=2g^{\mu\nu}\mathbb{I}_{n\times n}$ \\
Here $S^{\mu\nu}=\frac{i}{4}[\gamma^\mu,\gamma^\nu]$ gives the transformation
$L = \exp \left(-\frac{i}{2}\omega_{\mu\nu}J^{\mu\nu}\right)$. Represented by $\gamma^{j}=i\sigma^{j}$
in two dimensions. This works because $\{\gamma^i, \gamma^j \}=-2\delta^{ij}$
\\
 {\textbf{REPRESENTATIONS:}} \\
$t_r^a$ is, and can, always be chosen hermitian! \\
 {\textbf{Conjugate representation:}} \\
$t_{\overline{r}}^a = -(t_r^a)^* = -(t_r^a)^T$ \\
 {\textbf{Fundamental representation:}} \\
The lowest dimensional representation. For $\mathfrak{su}(n)$, $t_r^a$ is a $n\times n$ matrix.
For $n=2$ it is exactly $\sigma^a /2$. \\
For $\mathfrak{so}(n)$ it is strictly real $n\times n$ matrices.
\\
 {\textbf{Adjoint representation:}} \\
$(t_G^b)_{ac} \equiv if^{abc}$ where $[t^a,t^b]=if^{abc}t^c$. \\
Vector bosons are always in the adjoint representation. \\
$\frac{1}{i}(t_G^b)_{ac}=f^{abc}=\varepsilon^{abc}$ for $\mathfrak{su}(2)$.


















\subsection*{\small  WICK-PICTURE}
Coefficients of Fourier nodes are ladder operators (second quantization).
\includegraphics[width=0.2\textwidth]{WickContraction.png} \\
\includegraphics[width=0.3\textwidth]{Wick.png}

















\subsection*{\small  PATH INTEGRALS}
Diagrammatically, the relation between path integrals and the second quantized theory is
\[
\bra{...}\mathbb{T}\{...\}\ket{...} \longleftrightarrow \int \mathcal{D}\phi
\text{ and }
-\int dt H_I \longleftrightarrow  S=\int d^4x \mathcal{L}.
\]
An illustrational example is that of a scalar theory: \\
\includegraphics[width=0.27\textwidth]{PathAndWick.png} \\
This gives
\EQU{
&\bra{\Omega}\mathbb{T}\{\phi(x_1)...\phi(x_n)\} \ket{\Omega} \\
&=
\underset{T\rightarrow \infty(1-i\varepsilon)}{\text{lim}}
\frac{
\int \mathcal{D}\phi \phi(x_1)...\phi(x_n)\exp \left[i\int_{-T}^T\mathcal{L}d^4x \right]
}{
\int \mathcal{D}\phi \exp \left[i\int_{-T}^T\mathcal{L}d^4x \right]
}
}
\\
 {\textbf{Functional calculus:}} \\
A functional is a mapping from the space of functions to a real number
$F: D(\Omega) \rightarrow \mathbb{R}$ (or possibly $\mathbb{C}$) by $f\mapsto F[f]$.
A countinuous, linear functional is said to be a distribution. A "regular" distribution is one where there exists
a locally integrable function $\phi$ so that $F[f]=\int_{\Omega \subset \mathbb{R}}\phi(x)f(x)dx$
for all $f(x)\in D(\Omega)$. In that case
\[
\frac{\delta }{\delta f(y)}F[f]
= \frac{\delta }{\delta f(y)} \int \phi(x)f(x)dx \equiv \phi(y).
\]
More generally (in four dimensions)
\[
\frac{\delta }{\delta f(y)}f(x)=\delta^{(4)}(x-y).
\]
In this sense, the product rule and the chain rule still holds. Also
\[
\frac{\delta }{\delta f(x)} \int d^4 y (\partial_\mu f(y)) V^\mu(y) = -\partial_\mu V^\mu(x).
\]
\\
 {\textbf{Generating function:}}
For each independent field $\phi$, define a generating functional \\
\[
Z[J]=\int \mathcal{D}\phi \exp \left[i\int d^4 x \left(\mathcal{L}+J(x)\phi(x)\right)\right]
\]
where $J(x)$ is called a {\it source term}.
Generating functional for free scalar theory:
\[
Z[J] = Z[0]\exp \left[-\frac{1}{2}\int d^4 x \int d^4 y J(x)D_F(x-y)J(y)\right]
\]
The $\frac{1}{2}$ factor comes directly from the Lagrangian $\mathcal{L}=\frac{1}{2}\phi(-\partial^2-m^2)\phi$.
Generating functional for free Dirac theory:
\[
Z[\overline{\eta},\eta] = Z[0,0]\exp \left[-\int d^4 x \int d^4 y \overline{\eta}(x)S_F(x-y)\eta(y)\right]
\]













\subsection*{\small  GRASSMANN NUMBERS}
Defined by $\theta \eta = -\eta \theta$ which implies $\theta^2=0$ for all $\theta$. \\
 {\textbf{Integration:}}\\
Every Taylor expansion terminates after at the quadratic term: $f(\theta)=A+B\theta$. Then
\[
\int d\theta f(\theta) = \int d\theta (A+B\theta) \equiv B
\]
from wanting to preserve invariance under shift $\theta \mapsto \theta + \eta$.
Furthermore
\[
\int d\theta \int d\eta \eta \theta = +1.
\]
Note also that $(\theta \eta)^* = -\theta^* \eta^*$.
\\
 {\textbf{Differentiation:}}\\
Really the same as integration
\[
\frac{d}{d\theta}\theta \eta = \eta = -\frac{d}{d\theta}\eta\theta.
\]














\subsection*{\small  INTEGRALS}
 {\textbf{Integrals of commuting numbers:}}
\EQU{
& \left(\prod_k^N \int d\xi_k\right)\exp \left(-\xi_i B_{ij} \xi_j \right)
= \pi^{\frac{N}{2}}(\text{det} B)^{-\frac{1}{2}} \\
& \left(\prod_k^N \int d\xi_k\right)\xi_l \xi_m \exp \left(-\xi_i B_{ij} \xi_j \right)
= \text{const}\times (\text{det} B)^{-\frac{1}{2}} (B^{-1})_{lm}
}
\\
 {\textbf{Grassmann integrals:}}
\EQU{
& \int d \theta^* d\theta e^{-\theta^* b \theta} = b \\
& \left(\prod_i \int d\theta^*_i \theta_i\right) \exp \left(-\theta^*_i B_{ij}\theta_j\right)
= \text{det} B \\
& \left(\prod_i \int d\theta^*_i \theta_i\right)\theta_k \theta^*_l \exp \left(-\theta^*_i B_{ij}\theta_j\right)
= (\text{det} B)(B^{-1})_{kl} \\
}
















\subsection*{\small  CORRELATION FUNCTIONS}
Building blocks to describe (not only) interactions! \\
 {\textbf{Interaction picture:}}
If $H=H_0+H_1$ in Schr\"odinger picture then a state in the interaction picture is
$\ket{\Psi_I}=e^{iH_0t/\hbar}\ket{\Psi}$ and an operator $O_I=e^{iH_0t/\hbar}Oe^{-iH_0t/\hbar}$.
\\
 {\textbf{Correlation functions (Wick-picture):}} \\
\EQU{
&\bra{\Omega}\mathbb{T}\{\phi(x_1)...\phi(x_n)\}\ket{\Omega} \\
&=
\underset{T\rightarrow \infty(1-i\varepsilon)}{\text{lim}}
\frac{
\bra{0}\mathbb{T}\left\{\phi_I(x_1)...\phi_I(x_n)\exp \left[-i\int_{-T}^Tdt H_I(t) \right] \right\}\ket{0}
}{
\bra{0}\mathbb{T}\left\{\exp \left[-i\int_{-T}^Tdt H_I(t) \right] \right\}\ket{0}
}
}
Which amounts to finding all {\it connected} Feynman diagrams.


















\subsection*{\small  S-MATRIX}
\[
\bra{\mathbf{p}_1...\mathbf{p}_n}S \ket{\mathbf{k}_\mathcal{A}\mathbf{k}_\mathcal{B}}
=
\lim_{T\rightarrow \infty}
\braket{\underbrace{\mathbf{p}_1...\mathbf{p}_n}_{\text{time } T}}{\underbrace{\mathbf{k}_\mathcal{A}\mathbf{k}_\mathcal{B}}_{\text{time } -T}}
\]
We often use $S=\mathbb{I}+iT$. Then $\bra{...}iT\ket{...}\propto (2\pi)^4 \delta^{(4)}(\sum p_{in}-\sum p_{out})$.
Constant of proportionality is $i\mathcal{M}$ (amplitude).
The non-trivial part of the $S$ matrix can be written in terms of Feynman diagrams: \\

\EQU{
&\bra{\mathbf{p}_1...\mathbf{p}_n}iT\ket{\mathbf{p}_\mathcal{A}\mathbf{p}_\mathcal{B}} \\
&=
\underset{T\rightarrow \infty(1-i\varepsilon)}{\text{lim}}
_0\bra{\mathbf{p}_1...\mathbf{p}_n}
\mathbb{T}\left\{\exp \left[-i\int_{-T}^Tdt H_I(t) \right] \right\}
\ket{\mathbf{p}_\mathcal{A}\mathbf{p}_\mathcal{B}}_0
\bigg|_{\substack{\text{Connected} \\ \text{amputated}} }
}


















\subsection*{\small  CROSS SECTION}
Cross section $\sigma$ is the constant of proportionality:
\[
\text{Number of events} \propto \frac{N_\mathcal{A}N_\mathcal{B}}{A}.
\]
Generally we want to determine $\mathcal{P}=|\braket{\phi_1...\phi_n}{\phi_\mathcal{A}\phi_\mathcal{B}}|^2$.
It is then often convenient to write
$_{out}\bra{\phi_1...\phi_n} = \left( \prod_f \int \frac{d^3p_f}{(2\pi)^3} \frac{\phi_f(\mathbf{p}_f)}{\sqrt{2E_f}} \right) \ _{out}\bra{\mathbf{p}_1...\mathbf{p}_n}$
\\
 {\textbf{Differential cross section:}} \\
\resizebox{\columnwidth}{!}{
\[
d\sigma =
\frac{1}{ 2E_\mathcal{A} 2E_\mathcal{B} |v_\mathcal{A}-v_\mathcal{B}| }
\left(\prod_f \frac{d^3p_f}{(2\pi)^3}\frac{1}{2E_f}\right)
|\mathcal{M}|^2
(2\pi)^4 \delta^{(4)}(p_\mathcal{A}+p_\mathcal{B}-\sum p_f)
\]
}
where $|v_\mathcal{A}-v_\mathcal{B}|=
\left|\frac{\overline{k_\mathcal{A}^z}}{\overline{E_\mathcal{A}}}-\frac{\overline{k_\mathcal{B}^z}}{\overline{E_\mathcal{B}}}\right|$
is understood to have the constraints $\overline{k_\mathcal{A}^z}+\overline{k_\mathcal{B}^z}=\sum p_f^z$
and $\overline{E_\mathcal{A}}+\overline{E_\mathcal{B}}=\sum E_f$. \\
The cross section for two final-state particles: \\
\resizebox{\columnwidth}{!}{\[
\left(\frac{d\sigma}{d\Omega}\right)_{CM} =
\frac{1}{ 2E_\mathcal{A} 2E_\mathcal{B} |v_\mathcal{A}-v_\mathcal{B}| }
\frac{|\mathbf{p}_1|}{(2\pi)^2 4 E_{cm}}
|\mathcal{M}(p_\mathcal{A},p_\mathcal{B}\rightarrow p_1,p_2)|^2
\]}
The two-body phase space is given by $\int d\Pi_2=\int d(\cos \theta) \frac{1}{16\pi}\frac{2|\mathbf{p}_1|}{E_{cm}}$.
\\
When, in an $2\rightarrow 2$ process, all four masses are equal, then:
\[
\left(\frac{d\sigma}{d\Omega}\right)_{CM} = \frac{|\mathcal{M}|^2}{64\pi^2E^2_{cm}}.
\]
A $2\rightarrow 2$ process may also be expressed in terms of integration over Mandelstam variables:
\[
\sigma = \frac{1}{64\pi s}\frac{1}{\mathbf{p}_{1,cms}^2}\int_{t_{min}}^{t_{max}} dt |\mathcal{M}|^2
\]
where
\EQU{
t_{max} &= m_1^2 + m_3^2 -2E_1E_3 + 2 |\mathbf{p}_1| |\mathbf{p}_3| \\
t_{min} &= m_1^2 + m_3^2 -2E_1E_3 - 2 |\mathbf{p}_1| |\mathbf{p}_3| \\
}
 {\textbf{Differential decay rate:}} \\
\resizebox{\columnwidth}{!}{\[
d\Gamma =
\frac{1}{2m_\mathcal{A}}
\left(\prod_f \frac{d^3p_f}{(2\pi)^3}\frac{1}{2E_f}\right)
|\mathcal{M}(p_\mathcal{A}\rightarrow \{p_f\})|^2
(2\pi)^4 \delta^{(4)}(p_\mathcal{A}+p_\mathcal{B}-\sum p_f)
\]}
 {\textbf{Phase space:}} \\
For identical particles, remember to put a factor $\frac{1}{n!}$ in front!
\[
\int d\Pi_n = \left(\prod_{final \ f} \int \frac{d^3 p_f}{(2\pi)^3} \frac{1}{2E_f}\right)
(2\pi)^4 \delta^{(4)}\left(\sum_i p_i - \sum_f p_f\right).
\]
This makes it possible to write
\[
\frac{d\sigma}{d\Pi_n} = \frac{|\mathcal{M}|^2}{2E_\mathcal{A}2E_\mathcal{B}|v_\mathcal{A}-v_\mathcal{B}|}
\text{ and }
\frac{d\Gamma}{d\Pi_n} = \frac{|\mathcal{M}|^2}{2m_\mathcal{A}}.
\]
 {\textbf{Tricks:}} \\
$E_\mathcal{A}E_\mathcal{B}|v_\mathcal{A}-v_\mathcal{B}|
=|E_\mathcal{B}p^z_\mathcal{A}-E_\mathcal{A}p^z_\mathcal{B}|
=|\varepsilon_{\mu x y \nu}p^\mu _\mathcal{A} p^\nu _\mathcal{B}|$ \\
$\delta(E_f-E_i)=\frac{E_f}{|\mathbf{p}_f|}\delta(|\mathbf{p}_f|-|\mathbf{p}_i|)$ for equal masses. \\
$\sum_{polarizations} \epsilon_\mu ^* \epsilon_\nu \mapsto -g_{\mu\nu}$ in QED.












\subsection*{\small  INTERACTIONS}
For theories involving only scalars $\mu \phi^3$ and $\lambda \phi^4$ are the only allowed renormalizable interactions.
The $\mu$ has dimension $1$ where $\lambda$ is dimensionless. Spinor self-interactions are not allowed since $\psi^3$
(besides violating Lorentz invariance) has dimension $9/2$. Only allowable new interaction between spinors and scalars
is $g\overline{\psi}\psi \phi$. Adding vector fields we can have scalar couplings like $eA^\mu \phi \partial_\mu \phi^*$
and $e^2 |\phi|^2 A^2$. With spinors we have $e\overline{\psi}\gamma^\mu \psi A_\mu$ and self coupling:
$A^2(\partial_\mu A^\mu)$ and $A^4$.
\\
 {\textbf{Vertex factors:}} \\
If all momenta are assigned to be ingoing (this means they are all initial states). Remember that antiparticles
and particles are defined by anti-aligned and aligned momenta with the particle number flow.
\\
 {\textbf{Mandelstam variables:}} \\
$s=(p_1+p_2)^2$, $t=(p_1-p_3)^2$ and $u=(p_1-p_4)^2$.
When on shell: $s+u+t=\sum_i m_i^2$. \\
\includegraphics[width=0.25\textwidth]{Mandelstam.png}  \\
\includegraphics[width=0.15\textwidth]{Mandelstam2.png}
\includegraphics[width=0.08\textwidth]{Mandelstam3.png}















\subsection*{\small  QED}
 {\textbf{Definitions:}}
Tensor field $F_{\mu\nu} \equiv \partial_\mu A_\nu - \partial_\nu A_\mu$. \\
Fine structure constant $\alpha=e^2/4\pi$.
$A^\mu=(\Phi,\mathbf{A})$ where $\Phi=Q/4\pi r$. Maxwells equations
$\varepsilon^{\mu\nu\rho \sigma}\partial_\nu F_{\rho \sigma}=0$ and
$\partial_\mu F^{\mu \nu}=ej^{\nu}$ with $j^{\nu}=\overline{\psi}\gamma^\nu \psi$
current density and field tensor $F^{\mu \nu} = \partial_\mu A_\nu - \partial_\nu A_\mu$. \\
 {\textbf{Ward identity:}}
$k_\mu \mathcal{M}^\mu=0$ where $\mathcal{M}=\varepsilon_\mu\mathcal{M}^\mu$.
Can also define $\mathcal{M}_\mu=\int d^4 x \bra{f}j^{\mu} \ket{i}$.
Physical significance: Longitudal polarization of the photon which arises in the $\xi$-gauge is unphysical
and disappears from the $S$-matrix. \\
The Ward identity implies that:
$\sum_\varepsilon \varepsilon^{*\mu}\varepsilon^\nu \mapsto -g^{\mu\nu}$ (not mathematical equivalence).
Also: True for QED, but not for any theory!

















\subsection*{\small  FADEEV-POPOV PROCESS}
Insert identity into functional integral: \\
\[
1 = \int \mathcal{D}\alpha \delta(G(A^\alpha))\text{det}\left(\frac{\delta G(A^\alpha)}{\delta \alpha}\right)
\]
where $A^\alpha_\mu = A_\mu + \frac{1}{e}\partial_\mu \alpha$.
So that \\
\[\int \mathcal{D}A e^{iS[A]}
= \int \mathcal{D}\alpha \int \mathcal{D}A e^{iS[A]} \delta(G(A^\alpha))\text{det}\left(\frac{\delta G(A^\alpha)}{\delta \alpha}\right)\]
Now choose the gauge fixing $G(A)=\partial^\mu A_\mu(x)-\omega(x)$ and integrate over all possible
$\omega(x)$ with a gaussian weighting function $\exp\left( -i\int d^4 x \frac{\omega^2}{2\xi} \right)$.
This results in \\
$\tilde{D}_F^{\mu \nu}(k)=\frac{-i}{k^2}(g^{\mu\nu}-(1-\xi)\frac{k^\mu k^\nu}{k^2})$\\
in momentum space. We are free to choose $\xi$.
For example $\xi=0$ (Landau gauge) and $\xi=1$ (Feynman gauge).
\\
 {\textbf{Non-abelian case:}}
In the non-abelian case $\text{det}\left(\frac{\delta G(A^\alpha)}{\delta \alpha}\right)$ is not independent of the $A$'s.
We can, however, recognise the determinant as a functional integral of a grassmann gaussian
\[
\text{det}\left(\frac{1}{g}\partial^\mu D_\mu \right)
=
\int \mathcal{D}c \mathcal{D}\overline{c} \exp \left[ i\int d^4x \overline{c}(-\partial^\mu D_\mu)c \right].
\]
Hence this problem
can be solved by introducing a new field in the Lagrangian: \\
\[
\mathcal{L} \supset
\overline{c}^a (-\partial^\mu D_\mu^{ac})c^c.
\]
These are the Faddeev-Popov ghost fields exhibiting fermi-dirac statistics and being scalars (this is sad).















\subsubsection*{\small SPONTANEOUS SYMMETRY BREAKING (SSB)}
For a Lagrangian of a set of $N$ fields $\phi^a$, set $V(\phi^a)$ equal to the
terms that don't involve derivatives. Change the mass coefficient
$m^2 \rightarrow - \mu^2$. $\phi^a_0=(0,0,...,v)$ minimizes this field, and we
call this the \textit{vacuum expectation value} og $\phi^a$. Shift the fields by
$\phi^a \rightarrow \phi^a(x) = (\pi^k(x), v+ \sigma(x))$ where $k=1,...,N-1$.
Rewriting the Lagrangian gives $N-1$ massless fields, and one massive field $\sigma$.
Explicitly:
\EQU{
&\mathcal{L}_{LSM} =
\frac{1}{2}(\partial_\mu\pi^k)^2+\frac{1}{2}(\partial_\mu \sigma)^2
-\frac{1}{2}(2\mu^2)\sigma^2-\sqrt{\lambda}\mu\sigma^3\\
&-\sqrt{\lambda}\mu (\pi^k)^2 \sigma - \frac{\lambda}{4}\sigma^4
-\frac{\lambda}{2}(\pi^k)^2 \sigma^2-\frac{\lambda}{4}[(\pi^k)^2]^2.
}



\subsubsection*{\small GOLDSTONE'S THEOREM}
{\textbf{Global:}}
For every spontaneously broken continuous symmetry, there theory must contain a
massless particle. Before the symmetry breaking the system has symmetry group
$O(N)$, and after it has $O(N-1)$. $O(N)$ can rotate in $N(N-1)/2$  directions,
and $O(N-1)$ in $(N-1)(N-2)/2$, so the number of broken symmetries is the
difference: $N-1$.
\\
{\textbf{Local (gauge):}}


\subsubsection*{\small THE HIGGS MECHANISM}
Introduce a complex field with $\mathcal{L}=|D_\mu \phi|^2 -V(\phi)$ that obeys the symmetries
and shifts the vacuum away from zero. If we choose the Mexican hat potential
\[
V(\phi) = -\mu |\phi|^2 + \frac{\lambda}{2}|\phi|^4
\]
we get a vacuum expectation value of $\expe{\phi}=\phi_0 = \left(\frac{\mu^2}{\lambda}\right)^\frac{1}{2}$.
We then break the symmetry (SSB style), knowing that this is only possible if we have a massless
scalar particle, not possible with intact symmetry (SSB $\rightarrow$ Goldstone boson).
Here it is conventional to set $\phi = \phi_0 + \frac{1}{\sqrt{2}}(\phi_1+i\phi_2)$. We can then write
\EQU{
|D_{\mu} \phi|^2 &= \frac{1}{2} (\partial_{\mu} \phi_1)^2 + \frac{1}{2}(\partial_{\mu} \phi_2)^2\\
& + \sqrt{2}e \phi_0 A_{\mu} \partial^{\mu} \phi_2 + e^2 \phi_0^2 A_{\mu} A^{\mu}+...
}
The 3rd term gives the coupling of Goldstone and massive gauge, and the fourth
term is the mass of the gauge.
To diagonalize the mass matrix, pick unitary gauge so that $\phi$ is real-valued. Then $\phi_2$ is removed and
we are left with
\[
|D_{\mu} \phi|^2-V(\phi)
=
(\partial_\mu \phi)^2 + e^2\phi^2 A_\mu A^\mu -V(\phi).
\]
In this case, the Goldstone boson vanishes, but it still provides the pole in the vacuum
polarization amplitude. This is the Higgs mechanism. The kinetic energy term of
$\mathcal{L}$ for the vacuum expectation value of $\phi$
\begin{align*}
|D_{\mu} \phi|^2 &= \frac{1}{2} (\partial_{\mu} \phi_1)^2 + \frac{1}{2}(\partial_{\mu} \phi_2)^2\\
& + \sqrt{2}e \phi_0 A_{\mu} \partial^{\mu} \phi_2 + e^2 \phi_0^2 A_{\mu} A^{\mu}+...
\end{align*}
We can always introduce
$\phi(x)=U(x)\frac{1}{\sqrt{2}}
\left(\begin{matrix}
0 \\ v + h(x)
\end{matrix}\right)$
where $U(x)$ is some $SU(N)$ transformation and $h$ is the physicsal Higgs field.

















\subsection*{\small  NON-ABELIAN GAUGE THEORIES}
The Lagrangian of a renormalizable theory can contain no terms of mass dimension higher than 4.
The terms that don't involve derivatives are invariant under local gauge. To make derivatives invariant
we introduce
\[(D_\mu \phi)_a
= \partial_\mu \phi_a - ig A_\mu^b (t^b)_{ac} \phi_c
= \partial_\mu\phi_a + gf^{abc}A_\mu^b \phi_c\]
for any field multiplet $\phi$. The result (for fermions) is the Yang-Mills lagrangian. Here \\
$[D_\mu,D_\nu] \equiv -ig F_{\mu\nu}^a t^a$, \\
$F_{\mu\nu}^a = \partial_\mu A_\nu^a - \partial_\nu A_\mu^a + g f^{abc}A_\mu^b A_\nu^c$. \\










\subsection*{\small  GLASHOW-WEINBERG-SALAM THEORY (GWS)}
Gauge theory for $SU(2)\times U(1)$. General transformation is then
$\phi \mapsto e^{i\alpha^a \tau^a}e^{iY\beta}\phi$ for hypercharge $Y=Q-T^3=1/2$ and $\tau^a=\frac{1}{2}\sigma^a$.
This gives
$D_\mu \phi = (\partial_\mu -igA_\mu^a\tau^a - i \frac{1}{2}g'B_\mu)\phi$.
The vaccuum expectation value
$\expe{\phi}=\frac{1}{\sqrt{2}}
\left(\begin{matrix}
0 \\ v
\end{matrix}\right)$
is left invariant if $\alpha_1 = \alpha_2 = 0$ and $\alpha_3 = \beta$.

The covariant derivative is
\[
D_\mu \phi = \left( \partial_\mu - igA_\mu^a\tau^a - iYg'B_\mu \right) \phi
\]
for $Y=1/2$ this gives
\begin{align*}
W_{\mu}^{\pm}&= \frac{1}{\sqrt{2}}(A_{\mu}^1 \mp iA_{\mu}^2),     \ \ \ \ m_W = g \frac{v}{2}\\
Z_{\mu}^0 &= \frac{1}{\sqrt{g^2 + g'^2}}(gA_{\mu}^3 -g'B_{\mu}),  \ \ \ \ m_Z= \sqrt{g^2+g'^2}\frac{v}{2}\\
A_{\mu} &= \frac{1}{\sqrt{g^2+g'^2}}(g'A_{\mu}^3+gB_{\mu}),       \ \ \ \ m_A=0
\end{align*}
as mass eigenstates. We may write this in terms of the weak mixing angle $\theta_w$
\EQU{
\left(\begin{matrix}
Z^0 \\ A
\end{matrix}\right)
=
\left(\begin{matrix}
  \cos \theta_w   & -\sin \theta_w \\
  \sin \theta_w   &  \cos \theta_w \\
\end{matrix}\right)
\left(\begin{matrix}
A^3 \\ B
\end{matrix}\right)
\text{ with }
\begin{matrix}
\cos \theta_w = \frac{g}{\sqrt{g^2 +g'^2}}
\\
\sin \theta_w = \frac{g'}{\sqrt{g^2 +g'^2}}
\end{matrix}
}
We may then relate the two coupling constants by $e = g \sin \theta_w$. In this basis
\EQU{
D_\mu &= \partial_\mu
- i \frac{g}{\sqrt{2}}\left(W_\mu^+T^++W_\mu^-T^-\right)\\
&- i \frac{g}{\cos \theta_w } Z_\mu (T^3-Q \sin^2 \theta_w)
- ie A_\mu Q,
}
having defined $T^\pm = T^1 \pm i T^2$ and $t^a_{ij}=iT^a_{ij}$ so that $T^a$ are real and antisymmetric.
$T^3$ are the eigenvalues of $T^3$.
\\
\textbf{Fermionic coupling: }
Since $\overline{\psi}i\slashed{\partial}\psi = \overline{\psi}_R i\slashed{\partial}\psi_R + \overline{\psi}_L i\slashed{\partial}\psi_L$
Dirac fermions split into separate pieces (left- and right-handed fields) when they are massless. In GWS we assign
the left-handed fermion fields to doublets of $SU(2)$, while making the right-handed fermion fields singlets under
this group ($SU(2)_L \times U(1)_Y$). Once $T^3$ is chosen, then $Y$ is determined. Hence $Y$ may differ for the left- and right-handed fields.
Note that no mass is allowed if $\psi_L \leftrightarrow \psi_R$ is to be maintained as a symmetry. This is because
\[
m \overline{\psi}\psi = m( \overline{\psi}_L \psi_R + \overline{\psi}_R \psi_L )
\]
The right-handed neutrino $\nu_R$ has isospin $I^3=0$, hence $Y=Q$, but since $Q=0$ we have
$T^3=Q=Y=0$. Hence $\nu_R$ cannot couple to anything. Fermions get their mass from a term
\[
\mathcal{L} \supset -\lambda_e \overline{E}^j_{aL} \phi^j e_{aR}
\]
for electrons. $-\lambda_e$ is the Yukawa coupling. $\overline{E}$ is the left-handed electron multiplet,
$E_L=(\nu_L,e_L)^T$, in spinor representation ($j$), $\phi$ is the higgs (scalar field) in the spinor
representation and $e$ is the right-handed electron singlet.






















\subsection*{\small  FEYNMAN RULES}
\includegraphics[width=0.3\textwidth]{FeynRules1.png} \\
\includegraphics[width=0.25\textwidth]{FeynRules2.png} \\
\includegraphics[width=0.3\textwidth]{FeynRules3.png} \\
\includegraphics[width=0.2\textwidth]{FeynRules4.png} \\

For complex scalar QED: \\
\includegraphics[width=0.3\textwidth]{ComplexScalarQED.png} \\


Example of complete rules: (Yukawa theory)\\
\includegraphics[width=0.3\textwidth]{FeynCompleteYukawa.png} \\








\subsection*{\small  SOME RESULTS:}
\textbf{Breit-Wigner}: \\
$f(E)\propto \frac{1}{E-E_0+i\Gamma/2}
\approx \frac{1}{2E_\mathbf{p}(p^0-E_\mathbf{p}+i(m/E_\mathbf{p})\Gamma/2)}$ \\
$\sigma \propto \frac{1}{(E-E_0)^2+\Gamma^2/4}$ \\

\subsection*{\small COMPTON SCATTERING: $e^- \gamma \rightarrow e^- \gamma$}
If $\omega$ is the energy of the photon, $\omega'$ is the energy of the final photon the
\textbf{Klein-Nishima} formula is
\[
\frac{d \sigma}{d \cos \theta}
= \frac{\pi \alpha^2}{m^2} \left(\frac{\omega'}{\omega} \right)^2
\left[\frac{\omega'}{\omega} + \frac{\omega}{\omega'} - \sin^2 \theta \right]
\]
Thomson cross section for scattering of classical electromagnetic radiation by a free electron
\[
\frac{d \sigma}{d \cos \theta}
= \frac{\pi \alpha^2}{m^2}(1+\cos^2 \theta) \text{ for } \alpha=\frac{e^2}{4\pi \hbar c} = \frac{e^2}{4\pi}.
\]
Note also Compton's formula for shift in the photon wavelength
\[
\frac{1}{\omega'} + \frac{1}{\omega} = \frac{1}{m}(1-\cos \theta).
\]

\subsection*{\small EXAMPLE: $e^+e^- \rightarrow \mu^+ \mu^-$}
\begin{itemize}
\item Correlation = $\bar{v}^{s'}(p')(-ie\gamma^{\mu})u^s(p)\big( \frac{-ig_{\mu \nu}}{q^2} \big) \bar{u}^r(k) (-ie\gamma^{\nu})v^{r'}(k')$
\item Squared matrix element \\
      $|\mathcal{M}|^2 = \frac{e^4}{q^4}\big( \bar{v}(p')\gamma^{\mu}u(p) \bar{u}(p)\gamma^{\mu} v(p') \big) \big(\bar{u}(k) \gamma_{\mu} v(k') \bar{v}(k') \gamma_{\nu} u(k) \big)$.
\item Sum over spins $\sum_{s,s'} \bar{v}_a^{s'}(p')\gamma^{\mu}_{ab} u_b^s(p )\bar{u}^s_c(p)\gamma^{\mu}_{cd}v_d^{s'}(p')$
\item This gives for \\
      $\frac{1}{4} \sum|\mathcal{M}|^2 = \frac{e^4}{4q^4} tr\big[(\cancel{p}' - m_e)\gamma^{\mu}(\cancel{p}+m_e) \gamma^{\nu} \big] tr \big[ (\cancel{k}+m_{\mu}) \gamma_{\mu}(\cancel{k}'-m_{\mu})\gamma^{\nu} \big]$.
\item Use trace magic and set $m_e=0$ to get \\
      $\frac{1}{4} \sum |\mathcal{M}|^2 = \frac{8e^4}{q^4} \big[(p \cdot k)(p' \cdot k') + (p \cdot k')(p' \cdot k) + m_{\mu}^2(p \cdot p') \big] $.
\item Total cross section $\sigma_{total} = \frac{4 \pi \alpha^2}{3E_{cm}^2} \sqrt{1-\frac{m_{\mu}^2}{E^2}}\big( 1+ \frac{1}{2} \frac{m_{\mu}^2}{E^2} \big)$
\end{itemize}



\end{multicols*}




\end{document}
